% Options for packages loaded elsewhere
\PassOptionsToPackage{unicode}{hyperref}
\PassOptionsToPackage{hyphens}{url}
%
\documentclass[
  parskip,
  oneside]{scrreprt}
\author{}
\date{\vspace{-2.5em}}

\usepackage{amsmath,amssymb}
\usepackage{lmodern}
\usepackage{iftex}
\ifPDFTeX
  \usepackage[T1]{fontenc}
  \usepackage[utf8]{inputenc}
  \usepackage{textcomp} % provide euro and other symbols
\else % if luatex or xetex
  \usepackage{unicode-math}
  \defaultfontfeatures{Scale=MatchLowercase}
  \defaultfontfeatures[\rmfamily]{Ligatures=TeX,Scale=1}
\fi
% Use upquote if available, for straight quotes in verbatim environments
\IfFileExists{upquote.sty}{\usepackage{upquote}}{}
\IfFileExists{microtype.sty}{% use microtype if available
  \usepackage[]{microtype}
  \UseMicrotypeSet[protrusion]{basicmath} % disable protrusion for tt fonts
}{}
\makeatletter
\@ifundefined{KOMAClassName}{% if non-KOMA class
  \IfFileExists{parskip.sty}{%
    \usepackage{parskip}
  }{% else
    \setlength{\parindent}{0pt}
    \setlength{\parskip}{6pt plus 2pt minus 1pt}}
}{% if KOMA class
  \KOMAoptions{parskip=half}}
\makeatother
\usepackage{xcolor}
\IfFileExists{xurl.sty}{\usepackage{xurl}}{} % add URL line breaks if available
\IfFileExists{bookmark.sty}{\usepackage{bookmark}}{\usepackage{hyperref}}
\hypersetup{
  hidelinks,
  pdfcreator={LaTeX via pandoc}}
\urlstyle{same} % disable monospaced font for URLs
\usepackage{color}
\usepackage{fancyvrb}
\newcommand{\VerbBar}{|}
\newcommand{\VERB}{\Verb[commandchars=\\\{\}]}
\DefineVerbatimEnvironment{Highlighting}{Verbatim}{commandchars=\\\{\}}
% Add ',fontsize=\small' for more characters per line
\usepackage{framed}
\definecolor{shadecolor}{RGB}{248,248,248}
\newenvironment{Shaded}{\begin{snugshade}}{\end{snugshade}}
\newcommand{\AlertTok}[1]{\textcolor[rgb]{0.94,0.16,0.16}{#1}}
\newcommand{\AnnotationTok}[1]{\textcolor[rgb]{0.56,0.35,0.01}{\textbf{\textit{#1}}}}
\newcommand{\AttributeTok}[1]{\textcolor[rgb]{0.77,0.63,0.00}{#1}}
\newcommand{\BaseNTok}[1]{\textcolor[rgb]{0.00,0.00,0.81}{#1}}
\newcommand{\BuiltInTok}[1]{#1}
\newcommand{\CharTok}[1]{\textcolor[rgb]{0.31,0.60,0.02}{#1}}
\newcommand{\CommentTok}[1]{\textcolor[rgb]{0.56,0.35,0.01}{\textit{#1}}}
\newcommand{\CommentVarTok}[1]{\textcolor[rgb]{0.56,0.35,0.01}{\textbf{\textit{#1}}}}
\newcommand{\ConstantTok}[1]{\textcolor[rgb]{0.00,0.00,0.00}{#1}}
\newcommand{\ControlFlowTok}[1]{\textcolor[rgb]{0.13,0.29,0.53}{\textbf{#1}}}
\newcommand{\DataTypeTok}[1]{\textcolor[rgb]{0.13,0.29,0.53}{#1}}
\newcommand{\DecValTok}[1]{\textcolor[rgb]{0.00,0.00,0.81}{#1}}
\newcommand{\DocumentationTok}[1]{\textcolor[rgb]{0.56,0.35,0.01}{\textbf{\textit{#1}}}}
\newcommand{\ErrorTok}[1]{\textcolor[rgb]{0.64,0.00,0.00}{\textbf{#1}}}
\newcommand{\ExtensionTok}[1]{#1}
\newcommand{\FloatTok}[1]{\textcolor[rgb]{0.00,0.00,0.81}{#1}}
\newcommand{\FunctionTok}[1]{\textcolor[rgb]{0.00,0.00,0.00}{#1}}
\newcommand{\ImportTok}[1]{#1}
\newcommand{\InformationTok}[1]{\textcolor[rgb]{0.56,0.35,0.01}{\textbf{\textit{#1}}}}
\newcommand{\KeywordTok}[1]{\textcolor[rgb]{0.13,0.29,0.53}{\textbf{#1}}}
\newcommand{\NormalTok}[1]{#1}
\newcommand{\OperatorTok}[1]{\textcolor[rgb]{0.81,0.36,0.00}{\textbf{#1}}}
\newcommand{\OtherTok}[1]{\textcolor[rgb]{0.56,0.35,0.01}{#1}}
\newcommand{\PreprocessorTok}[1]{\textcolor[rgb]{0.56,0.35,0.01}{\textit{#1}}}
\newcommand{\RegionMarkerTok}[1]{#1}
\newcommand{\SpecialCharTok}[1]{\textcolor[rgb]{0.00,0.00,0.00}{#1}}
\newcommand{\SpecialStringTok}[1]{\textcolor[rgb]{0.31,0.60,0.02}{#1}}
\newcommand{\StringTok}[1]{\textcolor[rgb]{0.31,0.60,0.02}{#1}}
\newcommand{\VariableTok}[1]{\textcolor[rgb]{0.00,0.00,0.00}{#1}}
\newcommand{\VerbatimStringTok}[1]{\textcolor[rgb]{0.31,0.60,0.02}{#1}}
\newcommand{\WarningTok}[1]{\textcolor[rgb]{0.56,0.35,0.01}{\textbf{\textit{#1}}}}
\usepackage{graphicx}
\makeatletter
\def\maxwidth{\ifdim\Gin@nat@width>\linewidth\linewidth\else\Gin@nat@width\fi}
\def\maxheight{\ifdim\Gin@nat@height>\textheight\textheight\else\Gin@nat@height\fi}
\makeatother
% Scale images if necessary, so that they will not overflow the page
% margins by default, and it is still possible to overwrite the defaults
% using explicit options in \includegraphics[width, height, ...]{}
\setkeys{Gin}{width=\maxwidth,height=\maxheight,keepaspectratio}
% Set default figure placement to htbp
\makeatletter
\def\fps@figure{htbp}
\makeatother
\setlength{\emergencystretch}{3em} % prevent overfull lines
\providecommand{\tightlist}{%
  \setlength{\itemsep}{0pt}\setlength{\parskip}{0pt}}
\setcounter{secnumdepth}{5}
\newlength{\cslhangindent}
\setlength{\cslhangindent}{1.5em}
\newlength{\csllabelwidth}
\setlength{\csllabelwidth}{3em}
\newlength{\cslentryspacingunit} % times entry-spacing
\setlength{\cslentryspacingunit}{\parskip}
\newenvironment{CSLReferences}[2] % #1 hanging-ident, #2 entry spacing
 {% don't indent paragraphs
  \setlength{\parindent}{0pt}
  % turn on hanging indent if param 1 is 1
  \ifodd #1
  \let\oldpar\par
  \def\par{\hangindent=\cslhangindent\oldpar}
  \fi
  % set entry spacing
  \setlength{\parskip}{#2\cslentryspacingunit}
 }%
 {}
\usepackage{calc}
\newcommand{\CSLBlock}[1]{#1\hfill\break}
\newcommand{\CSLLeftMargin}[1]{\parbox[t]{\csllabelwidth}{#1}}
\newcommand{\CSLRightInline}[1]{\parbox[t]{\linewidth - \csllabelwidth}{#1}\break}
\newcommand{\CSLIndent}[1]{\hspace{\cslhangindent}#1}
\usepackage[greek, ngerman, main=english]{babel}
\usepackage[utf8]{inputenc}
\usepackage[T1]{fontenc}
\usepackage{lmodern}
\usepackage[onehalfspacing]{setspace}
\usepackage[left=2.50cm, right=2.50cm, top=2.50cm, bottom=2.50cm, bindingoffset=10mm, includehead, includefoot]{geometry}
\usepackage[headsepline]{scrlayer-scrpage}
\usepackage{url}
\usepackage[backend=biber, style=authoryear, giveninits=true, maxbibnames=99, uniquename=init, maxcitenames=2, hyperref=true, date=year]{biblatex}
\usepackage{xpatch}
\usepackage{csquotes}
\usepackage{amsmath}
\usepackage{listings}
\usepackage{booktabs}
\usepackage{longtable}
\usepackage{multirow}
\usepackage{rotating}
\usepackage{subfigure}
\usepackage{graphicx}
\usepackage{float}
\usepackage{acronym}
\usepackage{lipsum}
\usepackage{scrhack}
\emergencystretch=50pt
\clubpenalty = 10000
\widowpenalty = 10000
\displaywidowpenalty = 10000
\automark[section]{chapter}
\renewcommand*{\chaptermarkformat}{}
\renewcommand*{\sectionmarkformat}{}
\setkomafont{title}{\sffamily}
\setkomafont{disposition}{\usekomafont{title}}
\setkomafont{author}{\usekomafont{title}}
\setkomafont{date}{\usekomafont{title}}
\setkomafont{caption}{\sffamily\small}
\setkomafont{captionlabel}{\usekomafont{caption}\bfseries\small}
\setkomafont{pagehead}{\normalfont\scshape}
\ifLuaTeX
  \usepackage{selnolig}  % disable illegal ligatures
\fi

\begin{document}

\begin{titlepage}
\centering
    {\Large Ruprecht-Karls-Universität Heidelberg\\
        Fakultät für Biowissenschaften\\
        Bachelorstudiengang Molekulare Biotechnologie\\}

    {\vspace{\stretch{2}}}
    {\usekomafont{title}

    {\Huge Data Analysis Report}

        {\Huge Topic 2, group 1}

        {\Huge Supervisor: Carl Hermann}
        {\Huge Tutor: Wanjun Hu}

    }

    \vspace{\stretch{2}}
    {\Large Data Science Project SoSe 2022}

    \vspace{\stretch{2}}

    {\Large
        \begin{tabular}{rl}
            Autoreb & Ekin Ören, Yoana Onishtenko, Linh Trinh, Junona Sachov\\
            ? & ?\\
            Abgabetermin &20.07.2022\\
        \end{tabular}
    }

    \vspace{\stretch{1}}

\end{titlepage}

\tableofcontents

\renewcommand\abstractname{\Large Acknowledgments}
\begin{abstract}
Thank You
\end{abstract}

\renewcommand\abstractname{\Large Acknowledgments}
\begin{abstract}

Thank You --> #do we need that or should we just write like a paper abstract or no abstract ?

\end{abstract}

\hypertarget{introduction}{%
\chapter{Introduction}\label{introduction}}

\hypertarget{cancer-and-brca}{%
\section{Cancer and BRCA}\label{cancer-and-brca}}

--\textgreater{} cancer generall (quick def, why important for research
(fall numbern worldwide + pic of map with brca cases higlighted in dif
color), role of data analysis here)

*überleitung

--\textgreater{} Brca (prevalance general + map, where how often,
clinical stages, therapies, role of data analysis here)

*überleitung

Up until this day cancer remains as \ldots. . We all know cancer as a
term for a diseases in which abnormal cells divide without control and
possibly invade nearby tissues. In our project we want to also shine
light on the metabolic pathways since metabolic activities are altered
in cancer cells relative to normal cells. These reprogrammed activities
improve cellular fitness to provide a selective advantage during
tumorigenesis.

Breast cancer is the most common malignant tumor and the second capital
reason for cancer death among women worldwide. To highlight the
importance of this research we want to draw your attention to the fact
that for 2022 the American Cancer Society Estimated about 290,560 breast
cancer cases in the US. and about 43,780 death cases.

\hypertarget{hallmarks}{%
\section{Hallmarks}\label{hallmarks}}

--\textgreater{} quick def + who postulated + when + pic of hallmarks

\hypertarget{methabolic-pathways---ascendany-of-tumerigenesis}{%
\section{Methabolic pathways - ascendany of
tumerigenesis}\label{methabolic-pathways---ascendany-of-tumerigenesis}}

--\textgreater{} /quote imortance of mpw from lit --\textgreater{} why
provide a selective advantage and where + energy --\textgreater{}
regulation of diff stages + warburg effect

\hypertarget{project-outline}{%
\section{Project outline}\label{project-outline}}

--\textgreater{} pic (overview of the projecrs objectives which our
analysis aims to answer) --\textgreater{} how, which methods, reg
analysis, what use of algorithm in research \& treatment \& diagnosis
-Computational Tools (Gene Set Scoring,Gene Set Variation Analysis, Gene
Set Enrichment Analysis, UMAP, PCA ) - our analysis: Discriptive
Analysis, Correlation analysis, Pan Cancer, Focused Analysis

we wanted to especially emphasize the importance of the pan cancer
analysis since by comparing, e.g., the upregulation of specific
(metabolic) pathways between tumor different typer types, provides the
posibility to explore and make connection between the tumerigenesis of
thoses tumors which hasnt been made before since its not offensichtlich
form just the information u get from primary research

First we cleaned our provided datasets by removing NAs, followed by a
biotype filtering to further reduce the size of the datasets. After the
biotype filtering in the datasets only protein-coding (genexpressions?)
remained. Basen on our given data we first wanted to focus our
regression analysis on the predition on wheter or not a tumor will be
infiltarting or not. After doing a umap on the data we discovered that
most of the patients have primary tumor and the analysis would not be
that intering. Therefor, we decided to proceed with predicting what
pathology stage the patient will have beased on gene/pathway
expression????

\hypertarget{related-work}{%
\section{Related work}\label{related-work}}

if we have time maybe ???

\hypertarget{material-and-methods}{%
\chapter{Material and Methods}\label{material-and-methods}}

\hypertarget{materials}{%
\section{2.1 Materials}\label{materials}}

\hypertarget{data-sets}{%
\subsection{2.1.1 Data sets}\label{data-sets}}

\begin{itemize}
\tightlist
\item
  tcga + annot
\item
  5 cancers
\end{itemize}

say that they were provided, dimension before cleaning, what contains
(rows, columns), what did we use it for or like what work did we perform
with it What kind of data do we have?

\hypertarget{used-packages}{%
\subsection{Used Packages}\label{used-packages}}

show a table! - where from - what they do - what we used them for

\hypertarget{methods}{%
\section{2.2 Methods}\label{methods}}

\hypertarget{data-exploration}{%
\subsection{2.2.1 Data Exploration}\label{data-exploration}}

2.2.1.1 Cleaning \& Filtering

-remobed NAs (why \& how) + code chunk + function - biotype filtering
(why + how) + function + code chunk

2.2.1.2 Discriptive analysis

\begin{itemize}
\item
  what is this + what is is used for
\item
  what we did (violin, boxplot, barplot, histos, variance mean, PCA
  \ldots{} What else ????) + plots, Q: do we here need code chunks ?

  Results: what did we acchive with this, what does it say, ow can we
  use thius inforamtion + in dicussion if needed point out what is
  interersting and if this needs further research
\end{itemize}

2.2.2 Metabolic pathways

\begin{itemize}
\tightlist
\item
  how did we find and retrieved the pathways
\item
  where did we extract them form + how code chunk ?
\item
  for what + what did we d further with them
\end{itemize}

2.2.3 Gene set scoring

\begin{itemize}
\tightlist
\item
  what is this
\item
  what used for generally
\item
  how did we use it + code chunk + what did we get (probs in results,
  right?) + plotsss + volcano ?
\item
  quality control ???
\end{itemize}

\#\#\#\#\#\#where does venn diagrmm gooooo

\hypertarget{pan-cancer}{%
\subsection{pan cancer}\label{pan-cancer}}

\hypertarget{gsea-vs-gsva}{%
\subsection{GSEA vs GSVA}\label{gsea-vs-gsva}}

\#\#\#correlation analysis \#\#\#regression analysis

nowww resultsss and info impuut, go sleepyy sleep

\hypertarget{results}{%
\chapter{Results}\label{results}}

\hypertarget{tumor-types-are-showing-disting-clusters-in-umap}{%
\section{33 tumor types are showing disting clusters in
UMAP}\label{tumor-types-are-showing-disting-clusters-in-umap}}

hello world!

\hypertarget{blb-alsdjflaskdf-umap-of-some-tumortypes}{%
\section{blb alsdjflaskdf umap of some
tumortypes}\label{blb-alsdjflaskdf-umap-of-some-tumortypes}}

Figure generation. You can do it with knitr or with latex formatting.
This is knitr:

\begin{figure}
\includegraphics[width=9.44in]{figures/figure1} \caption{Title. Description}\label{fig:figure1}
\end{figure}

easier alternative: this is latex formatting. In Figure \ref{figure1}
you can see an UMAP. (+ label your figures, equations etc and then
reference with /\ref{})

\begin{figure}[htbp]
    \centering
    \includegraphics{figures/figure1.png}
    \caption[\textbf{Title}.]{\textbf{Title}. Description.}
    \label{figure1}
\end{figure}

\begin{center}\rule{0.5\linewidth}{0.5pt}\end{center}

\hypertarget{results-1}{%
\chapter{Results}\label{results-1}}

\hypertarget{data-cleaning-filtering}{%
\section{1. Data Cleaning \& Filtering}\label{data-cleaning-filtering}}

until what dim did we reduce it and filter it

\hypertarget{discriptive-analysis}{%
\section{2. Discriptive analysis}\label{discriptive-analysis}}

describing onjectivly what we can see in the plot and then in
discussion: interpret what the plot and maybe compare to sth ?

\hypertarget{gene-set-scoring}{%
\section{3. gene set scoring}\label{gene-set-scoring}}

\hypertarget{umaps}{%
\section{Umaps ??}\label{umaps}}

\hypertarget{volcano-plots}{%
\section{Volcano plots}\label{volcano-plots}}

\hypertarget{gsea-vs-gsva-1}{%
\section{GSEA vs GSVA}\label{gsea-vs-gsva-1}}

the plots we got and what its saying ? venn diagramm

\hypertarget{correlation-analysis}{%
\section{Correlation analysis}\label{correlation-analysis}}

\hypertarget{regression-analysis}{%
\section{Regression analysis}\label{regression-analysis}}

\hypertarget{discussion}{%
\chapter{Discussion}\label{discussion}}

\hypertarget{immune-pathways-are-significantly-upregulated-in-x}{%
\section{Immune pathways are significantly upregulated in
X}\label{immune-pathways-are-significantly-upregulated-in-x}}

\hypertarget{references}{%
\chapter{References}\label{references}}

\hypertarget{refs}{}
\begin{CSLReferences}{0}{0}
\end{CSLReferences}

\hypertarget{appendix}{%
\chapter{Appendix}\label{appendix}}

\hypertarget{plots}{%
\section{Plots}\label{plots}}

hello \#\# Code world

\begin{Shaded}
\begin{Highlighting}[]
\CommentTok{\#createn einer liste mit allen patienten in dfs sortiert nach krebs}
\NormalTok{cancers }\OtherTok{=} \FunctionTok{list}\NormalTok{();cancers }\OtherTok{=} \FunctionTok{vector}\NormalTok{(}\StringTok{\textquotesingle{}list\textquotesingle{}}\NormalTok{,}\FunctionTok{length}\NormalTok{(}\FunctionTok{table}\NormalTok{(tcga\_anno}\SpecialCharTok{$}\NormalTok{cancer\_type\_abbreviation)))}
\FunctionTok{names}\NormalTok{(cancers) }\OtherTok{=} \FunctionTok{names}\NormalTok{(}\FunctionTok{table}\NormalTok{(tcga\_anno}\SpecialCharTok{$}\NormalTok{cancer\_type\_abbreviation))}
\NormalTok{i}\OtherTok{=}\DecValTok{1}
\ControlFlowTok{for}\NormalTok{ (i }\ControlFlowTok{in} \DecValTok{1}\SpecialCharTok{:}\FunctionTok{length}\NormalTok{(cancers))\{}
\NormalTok{  cancers[[i]] }\OtherTok{=}\NormalTok{ tcga\_exp\_cleaned[,tcga\_anno}\SpecialCharTok{$}\NormalTok{cancer\_type\_abbreviation }\SpecialCharTok{==} \FunctionTok{names}\NormalTok{(cancers)[i]]}
\NormalTok{\}}
\CommentTok{\#function die einen krebstypen df und genesets als input nimmt und ein df mit pvalues ausgibt}
\NormalTok{enrichment }\OtherTok{=} \ControlFlowTok{function}\NormalTok{(expressiondata, }\AttributeTok{genesets =}\NormalTok{ genesets\_ids)\{}
\NormalTok{  ESmatrix }\OtherTok{=} \FunctionTok{sapply}\NormalTok{(genesets, }\AttributeTok{FUN =} \ControlFlowTok{function}\NormalTok{(x)\{}
\NormalTok{    ins }\OtherTok{=} \FunctionTok{na.omit}\NormalTok{(}\FunctionTok{match}\NormalTok{(x,}\FunctionTok{rownames}\NormalTok{(expressiondata)))}\CommentTok{\#indices der gene im aktuellen set}
\NormalTok{    outs }\OtherTok{=} \SpecialCharTok{{-}}\NormalTok{ins}\CommentTok{\#indices der gene nicht im aktuellen set}
    \CommentTok{\#gibt einen vektor der für jeden patienten den pval für das aktuelle gene enthält}
\NormalTok{    res }\OtherTok{=} \ConstantTok{NULL}
    \ControlFlowTok{for}\NormalTok{ (i }\ControlFlowTok{in} \DecValTok{1}\SpecialCharTok{:}\FunctionTok{ncol}\NormalTok{(expressiondata))\{}\CommentTok{\#testet für jeden patienten}
\NormalTok{      res[i] }\OtherTok{=} \FunctionTok{wilcox.test}\NormalTok{(expressiondata[ins,i],expressiondata[outs,i],}\StringTok{\textquotesingle{}two.sided\textquotesingle{}}\NormalTok{)}\SpecialCharTok{$}\NormalTok{p.value}
\NormalTok{    \}}
    \FunctionTok{return}\NormalTok{(res)}
\NormalTok{  \})}
  \FunctionTok{row.names}\NormalTok{(ESmatrix) }\OtherTok{=} \FunctionTok{colnames}\NormalTok{(expressiondata); }\FunctionTok{return}\NormalTok{(ESmatrix)}
\NormalTok{\}}
\NormalTok{pvalueslist }\OtherTok{=} \FunctionTok{lapply}\NormalTok{(cancers, enrichment)}\CommentTok{\#für die tests für jeden krebstypen durch}
\end{Highlighting}
\end{Shaded}

\begin{Shaded}
\begin{Highlighting}[]
\NormalTok{get\_top10pathways\_from\_pvalues }\OtherTok{=} \ControlFlowTok{function}\NormalTok{(df\_p\_values, length\_genesets) \{}
  
  \FunctionTok{require}\NormalTok{(ggplot2)}
  
\NormalTok{  results }\OtherTok{\textless{}{-}} \FunctionTok{list}\NormalTok{()}
    
\NormalTok{  df\_p\_values\_log10 }\OtherTok{\textless{}{-}} \SpecialCharTok{{-}}\FunctionTok{log10}\NormalTok{(}\FunctionTok{as.data.frame}\NormalTok{(df\_p\_values))}
    
\NormalTok{  mean\_pathway }\OtherTok{\textless{}{-}} \FunctionTok{as.data.frame}\NormalTok{(}\FunctionTok{apply}\NormalTok{(df\_p\_values\_log10, }\DecValTok{1}\NormalTok{, mean))}
  \FunctionTok{rownames}\NormalTok{(mean\_pathway) }\OtherTok{\textless{}{-}} \FunctionTok{rownames}\NormalTok{(df\_p\_values\_log10)}
  
\NormalTok{  ordered\_score }\OtherTok{\textless{}{-}}\NormalTok{ mean\_pathway[}\FunctionTok{order}\NormalTok{(}\SpecialCharTok{{-}}\NormalTok{mean\_pathway[ ,}\DecValTok{1}\NormalTok{]), }\DecValTok{1}\NormalTok{]}
\NormalTok{  top\_10 }\OtherTok{\textless{}{-}} \FunctionTok{data.frame}\NormalTok{(ordered\_score[}\DecValTok{1}\SpecialCharTok{:}\DecValTok{10}\NormalTok{])}
  \FunctionTok{colnames}\NormalTok{(top\_10) }\OtherTok{\textless{}{-}} \StringTok{"mean\_pathway"}
  
\NormalTok{  ordered\_names }\OtherTok{\textless{}{-}} \FunctionTok{order}\NormalTok{(}\SpecialCharTok{{-}}\NormalTok{mean\_pathway[ ,}\DecValTok{1}\NormalTok{])}
\NormalTok{  top\_10\_names }\OtherTok{\textless{}{-}}\NormalTok{ ordered\_names[}\DecValTok{1}\SpecialCharTok{:}\DecValTok{10}\NormalTok{]}
\NormalTok{  top\_10}\SpecialCharTok{$}\NormalTok{pathway\_names }\OtherTok{\textless{}{-}} \FunctionTok{row.names}\NormalTok{(mean\_pathway)[top\_10\_names]}
  
\NormalTok{  results[[}\DecValTok{1}\NormalTok{]] }\OtherTok{\textless{}{-}}\NormalTok{ top\_10}
  
\NormalTok{  results[[}\DecValTok{2}\NormalTok{]] }\OtherTok{\textless{}{-}} \FunctionTok{ggplot}\NormalTok{(}\AttributeTok{data =}\NormalTok{ top\_10, }\FunctionTok{aes}\NormalTok{(}\AttributeTok{x =}\NormalTok{ mean\_pathway, }\AttributeTok{y =} \FunctionTok{reorder}\NormalTok{(pathway\_names, mean\_pathway)))}\SpecialCharTok{+}
    \FunctionTok{geom\_bar}\NormalTok{(}\AttributeTok{stat =} \StringTok{"identity"}\NormalTok{)}\SpecialCharTok{+}
    \FunctionTok{coord\_cartesian}\NormalTok{(}\AttributeTok{xlim =}\FunctionTok{c}\NormalTok{(}\DecValTok{3}\NormalTok{, }\FloatTok{3.75}\NormalTok{))}\SpecialCharTok{+}
    \FunctionTok{labs}\NormalTok{(}\AttributeTok{title =} \FunctionTok{names}\NormalTok{(df\_p\_values),}
         \AttributeTok{x =} \StringTok{"mean p{-}value pathway"}\NormalTok{,}
         \AttributeTok{y =} \StringTok{"pathway name"}\NormalTok{)}
  
\NormalTok{  pathway\_size }\OtherTok{\textless{}{-}} \FunctionTok{order}\NormalTok{(}\SpecialCharTok{{-}}\NormalTok{mean\_pathway[ ,}\DecValTok{1}\NormalTok{])}
\NormalTok{  top\_10\_size }\OtherTok{\textless{}{-}}\NormalTok{ pathway\_size[}\DecValTok{1}\SpecialCharTok{:}\DecValTok{10}\NormalTok{]}
\NormalTok{  top\_10}\SpecialCharTok{$}\NormalTok{pathway\_size }\OtherTok{\textless{}{-}}\NormalTok{ length\_genesets[top\_10\_size]}
  
\NormalTok{  results[[}\DecValTok{3}\NormalTok{]] }\OtherTok{\textless{}{-}} \FunctionTok{ggplot}\NormalTok{(}\AttributeTok{data =}\NormalTok{ top\_10, }\FunctionTok{aes}\NormalTok{(}\AttributeTok{x =}\NormalTok{ mean\_pathway, }\AttributeTok{y =} \FunctionTok{reorder}\NormalTok{(pathway\_names,}
\NormalTok{                                                                          mean\_pathway)))}\SpecialCharTok{+}
    \FunctionTok{geom\_point}\NormalTok{(}\FunctionTok{aes}\NormalTok{(}\AttributeTok{size =}\NormalTok{ pathway\_size))}\SpecialCharTok{+}
    \FunctionTok{labs}\NormalTok{(}\AttributeTok{title =} \FunctionTok{names}\NormalTok{(df\_p\_values),}
         \AttributeTok{x =} \StringTok{"mean p{-}value pathway"}\NormalTok{,}
         \AttributeTok{y =} \StringTok{"pathway name"}\NormalTok{)}
  
  \FunctionTok{return}\NormalTok{(results)}
\NormalTok{\}}
\end{Highlighting}
\end{Shaded}


\end{document}
